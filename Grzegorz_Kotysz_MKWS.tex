\documentclass[12pt,english]{article}
\usepackage[T1]{fontenc}
\usepackage[utf8]{inputenc}
\usepackage[a4paper]{geometry}
\geometry{verbose,tmargin=2cm,bmargin=2cm,lmargin=2cm,rmargin=2cm}
\usepackage{array}
\usepackage{float}
\usepackage{multirow}
\usepackage{amstext}
\usepackage{graphicx}
\usepackage{listings} % enables source code to be pasted

\makeatletter
\makeatother

\usepackage{babel}
\begin{document}

\title{ISAT-CK7 in OpenFOAM}


\author{Grzegorz Kotysz}


\date{UPDATE}

\maketitle
\newpage{}

\tableofcontents{}

\newpage{}

\section{Introduction}

The goal of this paper is to reproduce tutorial case\footnote{Daniel Moëll, Siemens Industrial Turbomachinery AB/Fluid Mechanics Lund University, 2015-12-08, url: http://www.tfd.chalmers.se/\~{}hani/kurser/OS\_CFD\_2015/DanielMoell/Presentation\_ISAT-CK7.pdf, Accessed: 06.04.2018}, which was prepared by Daniel Moëll for the course "CFD with OpenSource Software". 

\section{Preparing tutorial case}

The first step of this paper is to create an adjusted tutorial case. The first step is to copy the already present tutorial case from the OpenFOAM directory. As the OpenFOAM v1712 is used in this paper, there are some differences compared to the aforementioned work. In order to copy the tutorial case, following command has to be executed while being in OpenFOAM bash:
\begin{lstlisting}
cp -r 
$FOAM_TUTORIALS/combustion/reactingFoam/laminar/counterFlowFlame2D
$WM_PROJECT_USER_DIR/run
\end{lstlisting}

The directiory where counterFlowFlame2D tutorial was changed in the newer version of OpenFOAM, hence the change from \textit{ras} to \textit{laminar}. Then, the chemkin directory must be created in counterFlowFlame2D
\begin{lstlisting}
cd $WM_PROJECT_USER_DIR/run/counterFlowFlame2D
mkdir chemkin
\end{lstlisting}

Now the appropriate case files have to be edited in order to use chemkin files. Following changes to the constant/thermophysicalProperties file need to be made:
\begin{lstlisting}
chemistryReader foamChemistryReader;
foamChemistryFile "$FOAM_CASE/constant/reactions";
foamChemistryThermoFiler "$FOAM_CASE/constant/thermo.compressible:'
\end{lstlisting}
to:
\begin{lstlisting}
chemistryReader chemkinReader;
CHEMKINFile "$FOAM_CASE/chemkin/drm19.dat";
CHEMKINThermoFile "$FOAM_CASE/chemkin/thermo30.dat";
\end{lstlisting}
NOTE: SOME CHANGES TO CHEMKIN FILES MIGHT BE NECESSARY [TO BE CHECKED]

As the aforementioned tutorial aims to simulate premixed flame, some changes to boundary conditions wll be made. Afterwards, the case will simulate counterflow of two premixed fuel-air mixtures with the temperature of 305$K$ with the velocity equal to 0.8$m/s$. The ignition will be implemented as the high temperature zone in the middle of the domain by the use of setFields command, so the appropriate setFieldsDict file must be created.

\end{document}